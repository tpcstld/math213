\documentclass[titlepage]{article}
\usepackage{amsmath}
\usepackage{amssymb}
\usepackage{fullpage}
\usepackage{graphicx}

\DeclareMathOperator{\arccot}{arccot}

\title{Assignment 3}
\date{\today}
\author{Jerry Jiang\\ TianQi Shi}

\begin{document}
\maketitle

\noindent
\section{Problem Set 1}
\subsection{Problem 1}
a) $y'' + 3y' - 10y = 0$
\begin{align*}
    &\lambda ^2 + 3 \lambda - 10 = 0
 \\ &(\lambda + 5)(\lambda -  2) = 0
 \\ &\therefore \lambda = 2, -5
\end{align*}

\noindent So $y = c_1 e^{2x} + c_2 e^{-5x}$.\\

\noindent b) $y'' + 2y' + 3y = 0$
\begin{align*}
    &\lambda ^2 + 2 \lambda + 3 = 0
 \\ &\lambda = \frac{-2 \pm \sqrt{2^2 - 4 * 1 * 3}}{2}
 \\ &\lambda = \frac{-2 \pm \sqrt{4 - 12}}{2}
 \\ &\lambda = \frac{-2 \pm i\sqrt{8}}{2}
 \\ &\lambda = -1 \pm i\sqrt{2}
\end{align*}

\noindent So $y = e^{-x}(c_1 \sin(\sqrt{2}x) + c_2 \cos(\sqrt{2}x))$.

\subsection{Problem 2}
$$y'' - 3y' = e^x + x$$

\noindent First find homogeneous solution.

\begin{align*}
    &\lambda^2 - 3\lambda = 0
 \\ &\lambda (\lambda - 3) = 0
 \\ &\lambda = 0, 3
\end{align*}

\noindent So $y_h = c_1e^{0x} + c_2e^{3x}$.

\noindent Second find particular solution. Guess $y_p = Ae^x + Bx + C$.

\noindent $C$ is part of homogeneous solution. Guess $y_p = Ae^x + Bx^2 + Cx$ instead.

\begin{align*}
  &y_p' = Ae^x + 2Bx + C
  \\ &y_p'' = Ae^x + 2B
\end{align*}

\begin{align*}
  &y'' - 3y' = e^x + x
  \\ &Ae^x + 2B - 3(Ae^x + 2Bx + C) = e^x + x
  \\ &(A- 3A)e^x - 6Bx + (2B - 3C) = e^x + x
\end{align*}

\noindent Find $A$.
\begin{align*}
  \\ &A - 3A = 1
  \\ &-2A = 1
  \\ &A = -\frac{1}{2}
\end{align*}

\noindent Find $B$.
\begin{align*}
  \\ &-6B = 1
  \\ &B = -\frac{1}{6}
\end{align*}

\noindent Find $C$.
\begin{align*}
  \\ &2B - 3C = 0
  \\ &-\frac{2}{6} = 3C
  \\ &C = -\frac{1}{9}
\end{align*}

\noindent So $y_p = -\frac{1}{2}e^x -\frac{1}{6}x^2 - \frac{1}{9}x$.

\noindent So $y = -\frac{1}{2}e^x -\frac{1}{6}x^2 - \frac{1}{9}x + c_1 + c_2e^{3x}$.

\section{Problem Set 2}
\subsection{Problem 1}

We can show that a set of equations are linearly dependent if there exists a set of non-zero coefficients for the terms such that the sum is equal to 0.\\

\noindent a) $\{0, e^{x^2} \cos(\arctan x)\}$
$$1(0) + 0(e^{x^2} \cos(\arctan x)) = 0$$

\noindent b) $\{e^x, e^{x+\pi}, \sin x\}$
$$ce^x + e^{x + \pi} + 0(\sin x) = 0$$
This statement holds where $c = -e^{\pi}.$\\

\noindent c) $\{e^{-x}, \sinh x, \cosh x\}$
$$\sinh x - \cosh x + e^{-x} = 0$$
We know this statement holds by expanding the hyperbolic functions.

\begin{align*}
  \sinh x - \cosh x + e^{-x} &= \frac{e^x - e^{-x}}{2} - \frac{e^x + e^{-x}}{2} + e^{-x}
  \\ &= -e^{-x} + e^{-x}
  \\ &= 0
\end{align*}

\noindent d) $\{2x^2 - 1, 5, 1 - x^2\}$
\begin{align*}
  & (2x^2 - 1) + 2(1 - x^2) - \frac{1}{5}(5)
  \\ &= 2x^2 - 1 + 2 - 2x^2 - 1
  \\ &= 0
\end{align*}

\subsection{Problem 2}

First, we solve the homogeneous case using the characteristic equation, $$\lambda^2 + 3\lambda - 2 = 0$$

\noindent Using the quadratic formula, $$\lambda = \frac{-3 \pm \sqrt{17}}{2}$$

\noindent Thus, $$y_h = c_1e^{\frac{-3 + \sqrt{17}}{2}} + c_2e^{\frac{-3 - \sqrt{17}}{2}}$$

\noindent For the particular solution, we use the method of undetermined coefficients. Derivatives of $e^x(x^2 + 1)$ gives us $\{x^2e^x, xe^x, e^x\}$, so we try
\begin{align*}
  y_p &= Ax^2e^x + Bxe^x + Ce^x
  \\ y_p' &= Ax^2e^x + (2A + B)xe^x + (B + C)e^x
  \\ y_p'' &= Ax^2e^x + (4A + B)xe^x + (2A + 2B +C)e^x
\end{align*}

\noindent Substituting into ODE,
\begin{align*}
  & Ax^2e^x + (4A + B)xe^x + (2A + 2B +C)e^x + 3(Ax^2e^x + (2A + B)xe^x + (B + C)e^x) - 2(Ax^2e^x + Bxe^x + Ce^x)
  \\ &= e^x((A + 3A - 2A)x^2 + (4A + B + 6A + 3B - 2B)x + (2A + 2B + C + 3B + 3C - 2C))
\end{align*}

\noindent Comparing coefficients with RHS $e^x(x^2 + 1)$, we get
\begin{align*}
  A &= \frac{1}{2}
  \\ B &= \frac{-5}{2}
  \\ C &= \frac{25}{4}
\end{align*}

\noindent Thus $y_p = e^x(\frac{1}{2}x^2 - \frac{5}{2}x + \frac{25}{4})$. So the general solution is: $$y = y_h + y_p = c_1e^{\frac{-3 + \sqrt{17}}{2}} + c_2e^{\frac{-3 - \sqrt{17}}{2}} + e^x(\frac{1}{2}x^2 - \frac{5}{2}x + \frac{25}{4})$$

\end{document}
