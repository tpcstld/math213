\documentclass[titlepage]{article}
\usepackage{amsmath} \usepackage{amssymb}
\usepackage{fullpage}
\usepackage{graphicx}

\DeclareMathOperator{\arccot}{arccot}

\title{Assignment 8}
\date{\today}
\author{Jerry Jiang\\ TianQi Shi}

\begin{document}
\maketitle

\noindent
\section*{Problem Set 1}
\subsection*{Question 1}
$$f(x) = x^2 \text{ from } -2 \text{ to } 2$$

First, we find $a_0$.

\begin{align*}
    a_0 &= \frac{1}{4}\int_{-2}^{2} x^2 dx
    \\ &= \frac{1}{2}\int_{0}^{2} x^2 dx
    \\ &= \frac{1}{2} \frac{8}{3}
    \\ &= \frac{4}{3}
\end{align*}

Next, we find $a_n$.

\begin{align*}
    a_n &= \frac{1}{2} \int_{-2}^{2} x^2 \cos\bigg(\frac{n\pi x}{2}\bigg) dx
    \\ &= \int_{0}^{2} x^2 \cos\bigg(\frac{n\pi x}{2}\bigg) dx
    \\ &= \frac{x^2 \sin(\frac{n\pi x}{2})}{\frac{n\pi}{2}} \bigg|_0^2 - \int_{0}^{2}  \frac{2x \sin(\frac{n\pi x}{2})}{\frac{n\pi}{2}} dx
    \\ &= \frac{2x^2 \sin(\frac{n\pi x}{2})}{n\pi} \bigg|_0^2 - \frac{4}{n\pi} \int_{0}^{2} x \sin\bigg(\frac{n\pi x}{2}\bigg) dx
    \\ &= - \frac{4}{n\pi} \bigg( -\frac{x \cos(\frac{n\pi x}{2})}{\frac{n\pi}{2}} \bigg|_0^2 + \frac{2}{n\pi} \int_{0}^{2} \cos\bigg(\frac{n\pi x}{2}\bigg) dx \bigg)
    \\ &= - \frac{4}{n\pi} \bigg( -\frac{4 (-1)^n}{n\pi} + \frac{2}{n\pi} \bigg( \frac{2}{n\pi} \sin\bigg(\frac{n\pi x}{2}\bigg) \bigg|_0^2 \bigg)
    \\ &= - \frac{4}{n\pi} \bigg( -\frac{4 (-1)^n}{n\pi} \bigg)
    \\ &= \frac{16 (-1)^n}{(n\pi)^2}
\end{align*}

Next, we find $b_n$.

\begin{align*}
    b_n &= \frac{1}{2} \int_{-2}^{2} x^2 \sin\bigg(\frac{n\pi x}{2}\bigg) dx
    \\ &= 0
\end{align*}

Therefore,

$$f(x) = \frac{4}{3} + \sum_1^{\infty} \frac{16 (-1)^n}{(n\pi)^2}\cos(\frac{n\pi x}{2})$$

This converges for all x.
\subsection*{Question 2}

Assuming $f(x)$ is periodic.

$$f(x) = e^{2x}$$

\begin{align*}
    c_n &= \frac{1}{2\pi} \int_{-\pi}^{\pi} e^{2x}e^{-inx} dx
    \\ &= \frac{1}{2\pi} \int_{-\pi}^{\pi} e^{(2 - in)x} dx
    \\ &= \frac{1}{2\pi} \frac{e^{(2 - in)x}}{2 - in} \bigg|_{-\pi}^{\pi}
    \\ &= \frac{1}{2\pi(2 - in)}\bigg(e^{(2 - in)\pi} - e^{-(2 - in)\pi}\bigg)
\end{align*}

Therefore,

$$f(x) = \sum_{n=-\infty}^{\infty} \frac{1}{2\pi(2 - in)}\bigg(e^{(2 - in)\pi} - e^{-(2 - in)\pi}\bigg) e^{inx}$$

\section*{Problem Set 2}
\subsection*{Question 1}
Suppose $f(x)$ is odd, we use first principles to get the derivative, $$f'(x)= \lim_{h\to0}\frac{f(x+h)-f(x)}{h}$$ To determine if the derivative is odd or even, we substitute $-x$,
\begin{align*}
    f'(-x) &= \lim_{h\to0}\frac{f(-x+h)-f(-x)}{h} \\
    &= \lim_{h\to0}\frac{-f(x-h)+f(x)}{h} \\
    &= \lim_{h\to0}\frac{f(x-h)-f(x)}{-h} \\
    &= \lim_{h\to0}\frac{f(x-h)-f(x)}{-h} \\
    &= f'(x)
\end{align*}
Thus the derivative of an odd function is even.

\subsection*{Question 2}
$$\cot x = \frac{1}{\tan x}$$
Cotangent is not piecewise continuous on $[0, \pi]$ as $\tan x$ has values of $0$ at $0$ and $\pi$. Since cotangent is the reciprocal of tangent, it will approach infinity where at these values, making the function not continuous.

\section*{Simulation}

Simulation code:

\begin{verbatim}
[y, fs] = audioread('recording.wav');
[y1, fs1] = audioread('azim90elev0.wav');

out1 = conv(y, y1(:,1));
out2 = conv(y, y1(:,2));

out = horzcat(out1, out2);

sound(out, fs);
\end{verbatim}

The code loads the wav file that we recorded, as well as one of the wav files that was given.
It then convolutes each side of the given stereo wav file with the recorded wav file.
Finally, it combines the two audio streams together, and plays the resultant sound.

We observed that the resultant sound preserves the content of the recording, but
the perceived direction that the sound is coming from has been changed. We could
change the direction that the sound was projected from by convoluting it with
a different impulse file.

See \texttt{sounds.zip} for the collection of resultant wav files that we created
through the simulation code.

\end{document}
