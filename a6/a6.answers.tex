\documentclass[titlepage]{article}
\usepackage{amsmath} \usepackage{amssymb}
\usepackage{fullpage}
\usepackage{graphicx}

\DeclareMathOperator{\arccot}{arccot}

\title{Assignment 6}
\date{\today}
\author{Jerry Jiang\\ TianQi Shi}

\begin{document}
\maketitle

\noindent
\section{Problem Set 1}
\subsection{Question 1}
First we find the equation for this graph. $$f(t) = t - (t-3)H(t-3) - 3H(t-3) + H(t-3) - H(t-4) + (-(t-4)+1)H(t-4) + (t-5)H(t-5) + 2(t-5)H(t-5)$$ Then we take the Laplace transform.
\begin{align*}
  L\{f(t)\} &= \frac{1}{s^2} - \frac{e^{-3s}}{s^2} - \frac{3e^{-3s}}{s} + \frac{e^{-3s}}{s} - \frac{e^{-4s}}{s} - \frac{e^{-4s}}{s^2} + \frac{e^{-4s}}{s} + \frac{e^{-5s}}{s^2} + 2\frac{e^{-5s}}{s^2} \\
  &= \frac{1-e^{-3s}-e^{-4s}+3e^{-5s}}{s^2} - \frac{2e^{-3s}}{s}
\end{align*}

\subsection{Question 2}
We can use the theorem for periodic functions, $$L\{f(t)\} = \frac{1}{1-e^{-sT}} \int_0^T f(t)e^{-st}dt$$ Substituting the function $f(t) = \sin(t)$ with period $T=2\pi$, we get $$L\{\sin(t)\} = \frac{1}{1-e^{-s2\pi}} \int_0^{2\pi} \sin(t)e^{-st}dt$$ Using integration by parts, we can simplify the integral.
\begin{align*}
  \int_0^{2\pi} \sin(t)e^{-st}dt &= \frac{e^{-st}\sin(t)}{-s}\Big|_0^{2\pi} + \frac{1}{s} \int_0^{2\pi} e^{-st}\cos(t)dt \\
  \int_0^{2\pi} \sin(t)e^{-st}dt &= \frac{e^{-st}\sin(t)}{-s}\Big|_0^{2\pi} + \frac{1}{s} \bigg( \frac{e^{-st}\cos(t)}{-s}\Big|_0^{2\pi} - \frac{1}{s} \int_0^{2\pi} e^{-st}\sin(t)dt \bigg) \\
  \bigg( 1+\frac{1}{s^2} \bigg) \int_0^{2\pi} \sin(t)e^{-st}dt &= \frac{e^{-st}\sin(t)}{-s}\Big|_0^{2\pi} + \frac{1}{s} \frac{e^{-st}\cos(t)}{-s}\Big|_0^{2\pi} \\
  \frac{s^2 + 1}{s^2} \int_0^{2\pi} \sin(t)e^{-st}dt &= \frac{1}{s} \bigg( \frac{e^{-s2\pi}}{-s} - \frac{e^{0}}{-s} \bigg) \\
  \int_0^{2\pi} \sin(t)e^{-st}dt &= \frac{1 - e^{-s2\pi}}{s^2 + 1}
\end{align*}
Substituting into the origin expression,
\begin{align*}
  L\{\sin(t)\} &= \frac{1}{1-e^{-s2\pi}} \int_0^{2\pi} \sin(t)e^{-st}dt \\
  &= \frac{1}{1-e^{-s2\pi}} \frac{1-e^{-s2\pi}}{s^2+1} \\
  &= \frac{1}{s^2 + 1}
\end{align*}
Therefore the Laplace transform of $\sin(t)$ is $\frac{1}{s^2 + 1}$.

\section{Problem Set 2}
\subsection{Question 1}
$$cos(4t), 0 < t < 2\pi$$

\noindent We know that

\begin{align*}
  L\{cos(4t)\} &= \frac{1}{1 - e^{-sT}} \int_0^T cos(4t)e^{-st} dt
  \\&= \frac{1}{1 - e^{-s2\pi}} \int_0^{2\pi} cos(4t)e^{-st} dt
\end{align*}

\noindent Use integration by parts to find the integral

\begin{align*}
  \int_0^{2\pi} \cos(4t)e^{-st} dt &= \frac{e^{-st}\cos(4t)}{-s}\Big|_0^{2\pi} - \frac{4}{s} \int_0^{2\pi} e^{-st}\sin(4t) dt
  \\&= \frac{e^{-st}\cos(4t)}{-s}\Big|_0^{2\pi} - \frac{4}{s} \bigg(\frac{e^{-st}\sin(4t)}{-s}\Big|_0^{2\pi} + \frac{4}{s}\int_0^{2\pi}e^{-st}\cos(4t) dt\bigg)
  \\\bigg(1 + \frac{16}{s^2}\bigg)\int_0^{2\pi} \cos(4t)e^{-st} dt &= \frac{e^{-st}\cos(4t)}{-s}\Big|_0^{2\pi} - \frac{4}{s}\frac{e^{-st}\sin(4t)}{-s}\Big|_0^{2\pi}
  \\\bigg(\frac{s^2 + 16}{s^2}\bigg)\int_0^{2\pi} \cos(4t)e^{-st} dt &= \frac{e^{-s2\pi}}{-s} - \frac{1}{-s} - 0 + 0
  \\&= \frac{1 - e^{-s2\pi}}{s}
  \\\int_0^{2\pi} \cos(4t)e^{-st} dt &= \frac{s(1 - e^{-s2\pi})}{s^2 + 16}
\end{align*}

\noindent Substitute back into the original equation

\begin{align*}
  \\L\{cos(4t)\} &= \frac{1}{1 - e^{-s2\pi}} \int_0^{2\pi} cos(4t)e^{-st} dt
  \\&= \frac{1}{1 - e^{-s2\pi}} * \frac{s(1 - e^{-s2\pi})}{s^2 + 16}
  \\&= \frac{s}{s^2 + 16}
\end{align*}

\noindent So the Laplace transform of $cos(4t)$ is $\frac{s}{s^2 + 16}$.

\subsection{Question 2}
Assuming $f(t) = 0$ when $t \le 0$ and $t \ge 4$. As the question is unclear.

$$x'' - x = f(t), x(0) = 0, x'(0) = 0, f(t) = t when 0 < t < 4$$

\noindent Turn into a heaviside function

$$x'' - x = t - 4u(t - 4) - (t - 4)u(t - 4)$$

\noindent Take the Laplace of both sides

\begin{align*}
    &x'' - x = t - 4u(t - 4) - (t - 4)u(t - 4)
 \\ &s^2X(s) - X(s) = \frac{1}{s^2} - \frac{4e^{-4s}}{s} - \frac{e^{-4s}}{s^2}
 \\ &(s^2 - 1)X(s) = \frac{1}{s^2} - \frac{4e^{-4s}}{s} - \frac{e^{-4s}}{s^2}
 \\ &X(s) = \frac{1}{s^2(s - 1)(s + 1)} - \frac{4e^{-4s}}{s(s - 1)(s + 1)} - \frac{e^{-4s}}{s^2(s - 1)(s + 1)}
\end{align*}

\noindent Use partial fraction decomposition.

\begin{align*}
    &\frac{1}{s^2(s - 1)(s + 1)} = \frac{A}{s} + \frac{B}{s^2} + \frac{C}{s - 1} + \frac{D}{s + 1}
 \\ &1 = As^3 - As + Bs^2 - B + Cs^3 + Cs^2 + Ds^3 - Ds^2
 \\ &A + C + D = 0, B + C - D = 0, -A = 0, -B = 1
 \\ &A = 0, B = -1, C = \frac{1}{2}, D = -\frac{1}{2}
\end{align*}

\begin{align*}
    &\frac{1}{s(s - 1)(s + 1)} = \frac{A}{s} + \frac{B}{s - 1} + \frac{C}{s + 1}
 \\ &1 = As^2 - A + Bs^2 + Bs + Cs^2 - Cs
 \\ &A + B + C = 0, B - C = 0, -A = 1
 \\ &A = -1, B = \frac{1}{2}, C = \frac{1}{2}
\end{align*}

\noindent Putting it all together

\begin{align*}
    &X(s) = \frac{1}{s^2(s - 1)(s + 1)} - \frac{4e^{-4s}}{s(s - 1)(s + 1)} - \frac{e^{-4s}}{s^2(s - 1)(s + 1)}
 \\ &X(s) = -\frac{1}{s^2} + \frac{1}{2}\frac{1}{s - 1} - \frac{1}{2}\frac{1}{s + 1}
    + 4e^{-4s}\bigg(-\frac{1}{s} + \frac{1}{2}\frac{1}{s - 1} + \frac{1}{2}\frac{1}{s + 1}\bigg)
    + e^{-4s}\bigg(-\frac{1}{s^2} + \frac{1}{2}\frac{1}{s - 1} - \frac{1}{2}\frac{1}{s + 1}\bigg)
 \\ &X(s) = -\frac{1}{s^2} + \frac{1}{2}\frac{1}{s - 1} - \frac{1}{2}\frac{1}{s + 1}
    + e^{-4s}\bigg(-\frac{4}{s} - \frac{1}{s^2} + \frac{5}{2}\frac{1}{s - 1} + \frac{3}{2}\frac{1}{s + 1}\bigg)
    \\ &x(t) = -t + \frac{e^{t}}{2} - \frac{e^{-t}}{2} + u(t - 4)\bigg(-4 - (t - 4) + \frac{5e^{t - 4}}{2} + \frac{3e^{-(t - 4)}}{2}\bigg)
\end{align*}
\end{document}
