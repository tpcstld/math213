\documentclass[titlepage]{article}

\usepackage{amsmath}
\usepackage{fullpage}

\title{Assignment 1}
\date{\today}
\author{Jerry Jiang\\ Dinah Shi}

\begin{document}
\maketitle

\noindent
\section{Problem Set 1}
\subsection{Problem 1}
a) $y' - 2y = 0$

\begin{align*}
  y' - 2y &= 0
  \\ Ne^{Nx} - 2e^{Nx} &= 0
  \\ Ne^{Nx} &= 2e^{Nx}
  \\ N = 2
\end{align*}
Therefore $N = 2$.

\vspace{1em}
\noindent
b) $y'' + 4y = 0$

\begin{align*}
  y'' + 4y &= 0
  \\ N^2e^{Nx} + 4e^{Nx} &= 0
  \\ N^2e^{Nx} &= -4e^{Nx}
  \\ N^2 &= -4
\end{align*}
Since $N^2 >= 0$, there are no such $N$'s.

\subsection{Problem 2}

a) First we shall find the derivatives of $y$.

\begin{align*}
  y &= Ncos(2x) + x
  \\ y' &= -2Nsin(2x) + 1
  \\ y'' &= -4Ncos(2x)
\end{align*}
Next we shall check if $y$ is a solution.
\begin{align*}
  &y'' + 4y
  \\ &= -4Ncos(2x) + 4(Ncos(2x) + x)
  \\ &= -4Ncos(2x) + 4Ncos(2x) + 4x
  \\ &= 4x
  \\ &= RHS
\end{align*}
Therefore, $y$ is a solution.

\vspace{1em}
\noindent
b)
\begin{align*}
  y'(x) &= -2Nsin(2x) + 1
  \\ y'(0) &= -2Nsin(0) + 1
  \\ 2 &= -2Nsin(0) + 1
  \\ 2 &= -2N0 + 1
  \\ 2 &= 1
\end{align*}
Therefore, there are no possible solutions.

\vspace{1em}
\noindent
c) Linear, as it can be written in the form
$a_n(x)y^{(n)}(x) + a_{n-1}(x)y^{(n-1)}(x) + ... + a_0(x)y(t) = g(x)$.
Where $n = 2$,
$a_2 = 1$,
$a_1 = 4$,
$a_0 = 0$, and
$g(x) = 4x$.

\noindent
\section{Problem Set 2}
\subsection{Problem 1}
Given $y'' = 2y + y'$, verify the following solutions:
\begin{align*}
y_1(x) &= sinh(2x) + cosh(2x)
\\ y_2(x) &= sin(2x+3)
\end{align*}

For $y_1$,
\begin{align*}
y_1(x) &= sinh(2x) + cosh(2x)
\\ y_1'(x) &= 2cosh(2x) + 2sinh(2x)
\\ y_1''(x) &= 4sinh(2x) + 4cosh(2x)
\end{align*}
Substituting into the differential equation,
$$4sinh(2x) + 4cosh(2x) = 2(sinh(2x) + cosh(2x)) + 2cosh(2x) + 2sinh(2x)$$
This statement holds so $y_1$ is a solution. For $y_2$,
\begin{align*}
y_2(x) &= sin(2x+3)
\\ y_2'(x) &= 2cos(2x+3)
\\ y_2''(x) &= 4cos(2x+3)
\end{align*}
Substituting into the differential equation,
$$4cos(2x+3) = 2sin(2x+3) + 2cos(2x+3)$$
This statement holds so $y_2$ is a solution.

\subsection{Problem 2}
Given the following differential equations:
$$y_1(x) = e^x + y' + y'' = 3$$
$$y_1(x) = e^y + y' + y'' = 3$$
Determine which equation is linear. For the linear equation, verify that
$$y_3(x) = Ae^{-x} + B + 3x - \frac{e^x}{2}$$
is a solution, for all constants $A$ and $B$. Then, determine one set of possible values for $A$ and $B$ given that $y_3(-1) = 2e - \frac{1}{2e}$.

We know $y_1$ is the linear equation as $y_2$ contains a term $e^y$ which is non-linear. To verify that $y_3$ is a solution,
\begin{align*}
y_3(x) &= Ae^{-x} + B + 3x - \frac{e^x}{2}
\\ y_3'(x) &= -Ae^{-x} + 3 - \frac{e^x}{2}
\\ y_3''(x) &= Ae^{-x} - \frac{e^x}{2}
\end{align*}
Substituting into the differential equation,
\begin{align*}
e^x + y' + y'' &= 3
\\ e^x + (-Ae^{-x} + 3 - \frac{e^x}{2}) + (Ae^{-x} - \frac{e^x}{2}) &= 3
\\ 3 &= 3
\end{align*}
This holds for all values of $A$ and $B$. Using the given parameters,
\begin{align*}
y_3(-1) &= 2e - \frac{1}{2e}
\\ 2e - \frac{1}{2e} &= Ae^{-(-1)} + B + 3(-1) - \frac{e^{-1}}{2}
\\ 2e - \frac{1}{2e} &= Ae + B - 3 - \frac{1}{2e}
\end{align*}
Thus we get $A = 2, B = 3$.
\end{document}
