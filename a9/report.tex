\documentclass[12pt]{article}

%amsmath is a packaged use for typesetting math
%amsfonts is required for special fonts, e.g. blackboard bold (for denoting real numbers, etc.)
\usepackage{amsmath,amsfonts}

\usepackage{fullpage,url,amssymb,epsfig,color,xspace}

\begin{document}
\begin{center}
{\Large\bf University of Waterloo}\\
\vspace{3mm}
{\Large\bf MATH 213, Spring 2015}\\
\vspace{2mm}
{\Large\bf Assignment 9}\\
\end{center}

\section*{Simulation}

The three included pictures \texttt{derv1.jpg}, \texttt{derv2.jpg}, and \texttt{derv3.jpg} show the derivation of the pressure equation. In summary,

$$p(x, t) = \sum_{n = 1}^{\infty} - \frac{8L}{(2n - 1)^2 \pi^2}
\cos\bigg(\frac{\pi (2n - 1)}{2L} x\bigg) \cos\bigg(\frac{\pi (2n - 1)}{2L} ct\bigg)$$

\noindent The length of the tube, $L$, that creates the 440Hz A note is
\begin{align*}
    L &= \frac{\lambda}{4} \\
      &= \frac{c}{4f} \\
      &= \frac{340.29}{4 * 440} \tag{c given by Google} \\
      &= 0.1933 \text{m} \tag{approx}
\end{align*}

\noindent See included \texttt{har[1, 5, 10, 20].wav} files for the sound of 1, 5, 10, 20 harmonics being retained, respectively.

\noindent See \texttt{solver.m} for the Matlab code used to generate the \texttt{.wav} files.

\end{document}
