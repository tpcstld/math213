% <- percent signs are used to comment
\documentclass[12pt]{article}

%amsmath is a packaged use for typesetting math
%amsfonts is required for special fonts, e.g. blackboard bold (for denoting real numbers, etc.)
\usepackage{amsmath,amsfonts}

\usepackage{fullpage,url,amssymb,epsfig,color,xspace}

%Note: Many of the packages above have other uses beyond those used in this document

%this marks the beginning of the document. Everything before this is called the Preamble.
\begin{document}
%marks the end of the title section
\begin{center}
{\Large\bf University of Waterloo}\\
\vspace{3mm}
{\Large\bf MATH 213, Spring 2015}\\
\vspace{2mm}
{\Large\bf Assignment 5}\\
\end{center}

\section*{Question 1}

Find the inverse of the given transform. \\

\noindent
a) $$\frac{3s^2 - 12s^2 - 2s - 56}{(s-7)s(s^2+4)}$$
b) $$\frac{s^4 + 3s^3 + 2s^2 + 27s + 18}{s^3(s^2+9)}$$

\section*{Question 2}

Use the Laplace transform to find the particular solution.

$$y''' - 2y'' + 4y' - 8y = 0$$
$$y(0) = 0, y'(0) = 0, y''(0) = 4$$

\noindent First, take the Laplace transform of the equation:

\begin{align*}
  &s^3Y(s) - s^2y(0) - sy'(0) - y''(0) - 2(s^2Y(s) - sy(0) - y'(0)) + 4(sY(s) - y(0)) - 8Y(s) = 0
  \\ &s^3Y(s) - 4 - 2s^2Y(s) + 4sY(s) - 8Y(s) = 0
  \\ &(s^3 - 2s^2 + 4s - 8)Y(s) = 4
  \\ &Y(s) = \frac{4}{s^3 - 2s^2 + 4s - 8}
  \\ &Y(s) = \frac{4}{(s-2)(x^2 + 4)} \tag{1 mark}
\end{align*}

\noindent Using partial fraction decomposition:
\begin{align*}
  &\frac{4}{(s-2)(x^2 + 4)} = \frac{A}{s-2} + \frac{Bs}{s^2 + 4} + \frac{2C}{s^2 + 4}
  \\ &4 = As^2 + 4A + Bs^2 - 2Bs + 2Cs - 4C
  \\ &A + B = 0
  \\ &-2B + 2C = 0
  \\ &4A - 4C = 4 \tag{1 mark}
\end{align*}

\noindent Solving for these equations gives $A = \frac{1}{2}$, $B = -\frac{1}{2}$, $C = -\frac{1}{2}$. (1 mark)

\noindent Therefore, (1 mark)

$$Y(s) = \frac{1}{2}*\frac{1}{s-2} - \frac{1}{2}*\frac{s}{s^2 + 4} - \frac{1}{2}*\frac{2C}{s^2 + 4}$$

\noindent Taking the inverse Laplace: (1 mark)

$$y(t) = \frac{1}{2}e^{2t} - \frac{1}{2}\cos(2t) - \frac{1}{2}\sin(2t)$$

\end{document}
