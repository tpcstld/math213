\documentclass[titlepage]{article}
\usepackage{amsmath} \usepackage{amssymb}
\usepackage{fullpage}
\usepackage{graphicx}

\DeclareMathOperator{\arccot}{arccot}

\title{Assignment 7}
\date{\today}
\author{Jerry Jiang\\ TianQi Shi}

\begin{document}
\maketitle

\noindent
\section*{Problem Set 1}
\subsection*{Question 1}

\subsubsection*{1.1}
For part 1.1, I'm going to assume you meant the inverse laplace, instead of the
laplace of the two rational expressions.

I shall prove this using induction, consider the base case: $n = 0$.
We know that $L\{t^0\} = L\{1\} = \frac{1}{s} = \frac{0!}{s^{0 + 1}}$.
So the statement is true for the base case.

Assume that the statement is true for some integer $k \ge 0$. Consider $k + 1$.

Consider the following equation.

$$Y(s) = \frac{(k+1)!}{s^{k+2}}$$

This is equivalent to

$$Y(s) = \frac{k!}{s^{k+1}}\frac{(k+1)}{s}$$

We know from the induction hypothesis that

$$L\{t^k\} = \frac{k!}{s^{k+1}}$$

So therefore

$$Y(s) = L\{t^k\}L\{k+1\}$$

Therefore, by convolution

$$y(t) = \int_{0}^{t}(k+1)(\tau^k) d\tau = t^{k+1}$$

So by the principle of mathematical induction,

$$L\{t^n\} = \frac{n!}{s^{n+1}}$$

\subsubsection*{1.2}
Also, again, assuming that you mean inverse laplace.

$$\frac{1}{(s+2)(s^2 + 1)} + \frac{3}{s^2 + 9}$$

Firstly, we know that

$$L^{-1}\{\frac{3}{s^2 + 9}\} = \sin(3t)$$
$$L^{-1}\{\frac{1}{s+2}\} = e^{-2t}$$
$$L^{-1}\{\frac{1}{s^2 + 1}\} = \sin(t)$$

So,

\begin{align*}
    L^{-1}\{\frac{1}{(s+2)(s^2 + 1)} + \frac{3}{s^2 + 9}\} &=
        \int_{0}^{t} \sin(\tau)e^{-2(t - \tau)} d\tau + \sin(3t)
        \\ &= e^{-2t}\int_{0}^{t} \sin(\tau)e^{2\tau} d\tau + \sin(3t)
\end{align*}

Now we solve the integral.

\begin{align*}
    \int_{0}^{t} \sin(\tau)e^{2\tau} d\tau
    & =\frac{e^{2\tau}}{2}\sin(\tau) \bigg|_0^t - \frac{1}{2}\int_{0}^{t} \cos(\tau)e^{2\tau} d\tau
    \\ &=\frac{e^{2\tau}}{2}\sin(\tau) \bigg|_0^t
    - \frac{1}{2}\bigg(\frac{e^{2\tau}}{2}\cos(\tau) \bigg|_0^t + \frac{1}{2}\int_{0}^{t} \sin(\tau)e^{2\tau} d\tau \bigg)
    \\ &=\frac{e^{2t}}{2}\sin(t)
    - \frac{1}{2}\bigg(\frac{e^{2t}}{2}\cos(t) - \frac{1}{2} + \frac{1}{2}\int_{0}^{t} \sin(\tau)e^{2\tau} d\tau \bigg)
    \\ 2 \int_{0}^{t} \sin(\tau)e^{2\tau} d\tau &= e^{2t}\sin(t)
    - \frac{e^{2t}}{2}\cos(t) + \frac{1}{2} - \frac{1}{2}\int_{0}^{t} \sin(\tau)e^{2\tau} d\tau
    \\ 4\int_{0}^{t} \sin(\tau)e^{2\tau} d\tau &= 2e^{2t}\sin(t)
    - e^{2t}\cos(t) + 1 - \int_{0}^{t} \sin(\tau)e^{2\tau} d\tau
    \\ 5\int_{0}^{t} \sin(\tau)e^{2\tau} d\tau &= 2e^{2t}\sin(t)
    - e^{2t}\cos(t) - 1
    \\ \int_{0}^{t} \sin(\tau)e^{2\tau} d\tau &= \frac{e^{2t}}{5}(2\sin(t) - \cos(t)) - \frac{1}{5}
\end{align*}

So therefore,

\begin{align*}
    L^{-1}\{\frac{1}{(s+2)(s^2 + 1)} + \frac{3}{s^2 + 9}\}
        &= e^{-2t}\int_{0}^{t} \sin(\tau)e^{2\tau} d\tau + \sin(3t)
    \\ &= e^{-2t}\bigg(\frac{e^{2t}}{5}(2\sin(t) - \cos(t)) - \frac{1}{5}\bigg) + \sin(3t)
    \\ &= \frac{1}{5}(2\sin(t) - \cos(t)) - \frac{e^{-2t}}{5} + \sin(3t)
\end{align*}

\subsection*{Question 2}

$$x'' + x' - 6x = 10 - u(t - 4), x(0) = x'(0) = 0$$

First we take the Laplace of both sides
\begin{align*}
    s^2X(s) + sX(s) - 6X(s) &= \frac{10 - e^{-4s}}{s}
    \\ X(s) &= \frac{10 - e^{-4s}}{s(s^2 + s - 6)}
    \\ &= \frac{10 - e^{-4s}}{s(s+3)(s-2)}
\end{align*}

Consider $\frac{1}{s(s+3)(s-2)}$.

\begin{align*}
    L^{-1}\{\frac{1}{s(s+3)(s-2)}\} = L^{-1}\{\frac{1}{s}\} * L^{-1}\{\frac{1}{s+3}\} * L^{-1}\{\frac{1}{s-2}\}
\end{align*}

Consider $L^{-1}\{\frac{1}{s}\} * L^{-1}\{\frac{1}{s+3}\}$.

\begin{align*}
    L^{-1}\{\frac{1}{s}\} * L^{-1}\{\frac{1}{s+3}\} &= \int_0^t e^{-3\tau} d\tau
    \\ &= -\frac{e^{-3t}}{3} + \frac{1}{3}
\end{align*}

We use this value in the second convolution equation.

\begin{align*}
    L^{-1}\{\frac{1}{s}\} * L^{-1}\{\frac{1}{s+3}\} * L^{-1}\{\frac{1}{s-2}\}
    &= \frac{1}{3}\int_0^t (1 - e^{-3\tau})e^{2t - 2\tau} d\tau
    \\ &= \frac{e^{2t}}{3}\int_0^t e^{-2\tau} - e^{-5\tau} d\tau
    \\ &= \frac{e^{2t}}{3} \bigg(-\frac{e^{-2\tau}}{2} + \frac{e^{-5\tau}}{5}\bigg) \bigg|_0^t
    \\ &= \frac{e^{2t}}{3} \bigg(-\frac{e^{-2t}}{2} + \frac{e^{-5t}}{5} + \frac{3}{10}\bigg)
    \\ &= \frac{1}{3} \bigg(-\frac{1}{2} + \frac{e^{-3t}}{5} + \frac{3e^{2t}}{10}\bigg)
\end{align*}

So in total, the final solution is.

$$x(t) = \frac{10}{3}\bigg(-\frac{1}{2} + \frac{e^{-3t}}{5} + \frac{3e^{2t}}{10}\bigg) - \frac{u(t - 4)}{3}\bigg(-\frac{1}{2} + \frac{e^{-3(t - 4)}}{5} + \frac{3e^{2(t - 4)}}{10}\bigg)$$

\section*{Problem Set 2}
\subsection*{Question 1}
\subsection*{Question 2}
\end{document}
