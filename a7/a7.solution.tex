% <- percent signs are used to comment
\documentclass[12pt]{article}

%amsmath is a packaged use for typesetting math
%amsfonts is required for special fonts, e.g. blackboard bold (for denoting real numbers, etc.)
\usepackage{amsmath,amsfonts}

\usepackage{fullpage,url,amssymb,epsfig,color,xspace}

%Note: Many of the packages above have other uses beyond those used in this document

%this marks the beginning of the document. Everything before this is called the Preamble.
\begin{document}
%marks the end of the title section
\begin{center}
{\Large\bf University of Waterloo}\\
\vspace{3mm}
{\Large\bf MATH 213, Spring 2015}\\
\vspace{2mm}
{\Large\bf Assignment 7}\\
\end{center}

\section*{Question 1 (7 marks)}
Find the inverse of the given transform in two different ways: using partial fractions and using the convolution theorem. $$F(s) = \frac{7}{(s-3)s^3}$$

\subsection*{Partial Fraction Expansion}
\begin{align*}
  \frac{7}{(s-3)s^3} &= \frac{A}{s-3} + \frac{B}{s^3} + \frac{C}{s^2} + \frac{D}{s} \\
  7 &= As^3 + B(s-3) + Cs(s-3) + Ds^2(s-3) \\
  7 &= (A+D)s^3 + (C-3D)s^2 + (B-3C)s - 3B
\end{align*}
Comparing coefficients, we get a system of equations:
\begin{align*}
  A+D &= 0 \\
  C-3D &= 0 \\
  B-3C &= 0 \\
  -3B &= 7
\end{align*}
We get $A=\frac{7}{27},B=\frac{-7}{3},C=\frac{-7}{9},D=\frac{-7}{27}$. So the original expression becomes, $$\frac{7}{(s-3)s^3}=\frac{7}{27}\bigg(\frac{1}{s-3}\bigg) - \frac{7}{3}\frac{1}{s^3} - \frac{7}{9}\frac{1}{s^2} - \frac{7}{27}\frac{1}{s}$$ Taking the inverse Laplace transform, $$f(t)=\frac{7}{27}e^{3t} - \frac{7}{6}t^2 - \frac{7}{9}t - \frac{7}{27}$$

\subsection*{Convolution Theorem}

We separate $F(s)$ into the following equation:

$$7 * \frac{1}{s^3} * \frac{1}{s-3}$$

\noindent Now we do the convolution:
\begin{align*}
    f(t) &= 7 \int_0^{t} \frac{1}{2}\tau^2e^{3(t - \tau)} d\tau
      \\ &= \frac{7e^{3t}}{2} \int_0^t \tau^2e^{-3\tau} d\tau
      \\ &= \frac{7e^{3t}}{2}
        \bigg(\frac{\tau^2e^{-3\tau}}{-3} \bigg|_0^t
        + \frac{2}{3} \int_0^t \tau e^{-3\tau} d\tau \bigg)
      \\ &= \frac{7e^{3t}}{2} \bigg(-\frac{t^2e^{-3t}}{3}
        + \frac{2}{3} \bigg(-\frac{\tau e^{-3\tau}}{3} \bigg|_0^t
        + \frac{1}{3} \int_0^t e^{-3\tau} d\tau \bigg) \bigg)
      \\ &= \frac{7e^{3t}}{2} \bigg(-\frac{t^2e^{-3t}}{3}
        + \frac{2}{3} \bigg(-\frac{te^{-3t}}{3}
        - \frac{1}{9} e^{-3\tau} \bigg|_0^t \bigg) \bigg)
      \\ &= \frac{7e^{3t}}{2} \bigg(-\frac{t^2e^{-3t}}{3}
        + \frac{2}{3} \bigg(-\frac{te^{-3t}}{3}
        - \frac{1}{9} e^{-3t} + \frac{1}{9} \bigg) \bigg)
      \\ &= -\frac{7t^2}{6}
        + \frac{7e^{3t}}{3} \bigg(-\frac{te^{-3t}}{3}
        - \frac{1}{9} e^{-3t} + \frac{1}{9} \bigg)
      \\ &= -\frac{7t^2}{6} -\frac{7t}{9}
        - \frac{7}{27} + \frac{7e^{3t}}{27}
\end{align*}

\section*{Question 2 (3 marks)}
Solve for $x(t)$ on $0 \leq t < \infty$, $$x''-7x'=\delta(t-11)$$

\noindent Taking the Laplace transform of the equation,
\begin{align*}
  s^2x(s) - 7sx(s) &= e^{-11s} \\
  x(s) &= \frac{e^{-11s}}{s(s-7)}
\end{align*}
Using partial fraction expansion,
\begin{align*}
  \frac{1}{s(s-7)} &= \frac{A}{s} + \frac{B}{s-7} \\
  A(s-7) + Bs &= 1
\end{align*}
Comparing coefficients, $$A=\frac{-1}{7}, B=\frac{11}{7}$$ Substituting back into the original expression, $$\frac{e^{-11s}}{s(s-7)} = e^{-11s}\bigg(\frac{1}{7s} - \frac{1}{7(s-7)}\bigg)$$ Taking the inverse Laplace transform, $$x(t)=\frac{-1}{7}u(t-11) + \frac{1}{7}e^{7(t-11)}u(t-11)$$

\end{document}
