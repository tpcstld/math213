% <- percent signs are used to comment
\documentclass[12pt]{article}

%amsmath is a packaged use for typesetting math
%amsfonts is required for special fonts, e.g. blackboard bold (for denoting real numbers, etc.)
\usepackage{amsmath,amsfonts}

\usepackage{fullpage,url,amssymb,epsfig,color,xspace}

%Note: Many of the packages above have other uses beyond those used in this document

%this marks the beginning of the document. Everything before this is called the Preamble.
\begin{document}
%marks the end of the title section
\begin{center}
{\Large\bf University of Waterloo}\\
\vspace{3mm}
{\Large\bf MATH 213, Spring 2015}\\
\vspace{2mm}
{\Large\bf Assignment 7}\\
\end{center}

\section*{Question 1}

\subsection*{Convolution Theorem}

We separate $F(s)$ into the following equation:

$$7 * \frac{1}{s^3} * \frac{1}{s-3}$$

\noindent Now we do the convolution:

\begin{align*}
    f(t) &= 7 \int_0^{t} \frac{1}{2}\tau^2e^{3(t - \tau)} d\tau
      \\ &= \frac{7e^{3t}}{2} \int_0^t \tau^2e^{-3\tau} d\tau
      \\ &= \frac{7e^{3t}}{2}
        \bigg(\frac{\tau^2e^{-3\tau}}{-3} \bigg|_0^t
        + \frac{2}{3} \int_0^t \tau e^{-3\tau} d\tau \bigg)
      \\ &= \frac{7e^{3t}}{2} \bigg(-\frac{t^2e^{-3t}}{3}
        + \frac{2}{3} \bigg(-\frac{\tau e^{-3\tau}}{3} \bigg|_0^t
        + \frac{1}{3} \int_0^t e^{-3\tau} d\tau \bigg) \bigg)
      \\ &= \frac{7e^{3t}}{2} \bigg(-\frac{t^2e^{-3t}}{3}
        + \frac{2}{3} \bigg(-\frac{te^{-3t}}{3}
        - \frac{1}{9} e^{-3\tau} \bigg|_0^t \bigg) \bigg)
      \\ &= \frac{7e^{3t}}{2} \bigg(-\frac{t^2e^{-3t}}{3}
        + \frac{2}{3} \bigg(-\frac{te^{-3t}}{3}
        - \frac{1}{9} e^{-3t} + \frac{1}{9} \bigg) \bigg)
      \\ &= -\frac{7t^2}{6}
        + \frac{7e^{3t}}{3} \bigg(-\frac{te^{-3t}}{3}
        - \frac{1}{9} e^{-3t} + \frac{1}{9} \bigg)
      \\ &= -\frac{7t^2}{6} -\frac{7t}{9}
        - \frac{7}{27} + \frac{7e^{3t}}{27}
\end{align*}

\subsection*{Partial Fraction Expansion}

Decompose the rational equation.

\begin{align*}
    \frac{7}{s^3(s-3)} &= \frac{A}{s} + \frac{B}{s^2} + \frac{C}{s^3} + \frac{D}{s-3}
    \\ 7 &= A(s^3 - 3s^2) + B(s^2 - 3s) + C(s - 3) + Ds^3
    \\ & A + D = 0, -3A + B = 0, -3B + C = 0, -3C = 7
    \\ \therefore &C = \frac{-7}{3}, B = \frac{-7}{9}, A = \frac{-7}{27}, D = \frac{7}{27}
    \\ \therefore \frac{7}{s^3(s-3)} &= \frac{-7}{27}\frac{1}{s} + \frac{-7}{9}\frac{1}{s^2} + \frac{-7}{3}\frac{1}{s^3} + \frac{7}{27}\frac{1}{s-3}
          \\ L\{\frac{7}{s^3(s-3)}\} &= \frac{-7}{27} + \frac{-7}{9}t + \frac{-7}{6}t^2 + \frac{7}{27}e^{3t}
\end{align*}

\section*{Question 2}

\end{document}
